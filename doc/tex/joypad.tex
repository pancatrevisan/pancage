\chapter{Joypad}
	\section{Introduction}
		The Joypad class acts just like an joypad. It have some buton variables thar can be active or inactive. The character used this states to update.
		
		All buttons have public access to simplify.
		
		When a key is typed, the event handler (java.awt.event.KeyListener) verifies the key and store the status in ``Hashtable$<$Integer, Integer$>$ buttonStates''. If the state is BUTTON\_TYPED the action \textbf{must} set it to BUTTON\_OFF in it's update method. Just use the ``Action.cleanInput'' method in the ``Action.update'' after doing the update.
		
		
		The Joypad.buttonBuffer is a buffer with buttons in the order that are pressed (or  threated).
	\section{Buttons}
		The buttons have 3 possible states:
		\begin{itemize}
			\item \textbf{public static final int BUTTON\_OFF = 0}: the button is released.
			\item \textbf{public static final int BUTTON\_TYPED = 1}: the button was typed (pressed one time and released).
			\item \textbf{public static final int BUTTON\_HOLD = 2}: the button was pressed and hold.
		\end{itemize}
		If the button status = \textbf{BUTTON\_TYPED} the character \emph{update} method must set it to \textbf{BUTTON\_OFF} after the use.
		
	\section{Attributes}
	
	\paragraph{Hashtable$<$String, Integer$>$ buttonCodes: } The name of buttons of controller, and the key code associated with it. The Action uses this map to update it state. For example: `` `BUTTON\_UP',67'', will map the button up to the key of asc CODE 67. 
	
	The Joypad can have as many buttons as the user need, but the button names must be accordling to the Action.necessaryButtons. For example, Action x have $necessaryButtons = [``BUTTON\_UP'', ``BUTTON\_X'']$ then the Joypad.buttonCodes must have the key maps for ``BUTTON\_UP'' and ``BUTTON\_X''.
	
	
	
	\section{Joypad Templates}
	Template creates a Joypad with some buttons.
	
	\begin{table}[H]
    \caption{Possible collisions with the scenary.}
    \label{tab:scenaryCollisionTypes}
    \centering
	
    \begin{tabular}{|l|p{5cm}|}
		\hline
		\textbf{Template} 				& \textbf{Buttons} \\
		\hline
		TEMPLATE\_JOYPAD\_10\_BUTTONS 	&  UP,  DOWN, LEFT,  RIGHT,  PAUSE,  A,  B,  C,  D, SPECIAL \\
		\hline
	\end{tabular}
	\end{table}
	
	